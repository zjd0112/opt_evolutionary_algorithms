%
% GNU courseware, XIN YUAN, 2017
%

\section{免疫算法}

\frame{
\centerline{\textbf{\Huge{免疫算法}}}
}

\frame{\frametitle{定义}

\begin{block}{定义}
一种具有生成+检测 (generate and test) 的迭代过程的搜索算法。
\end{block}
}

\frame{\frametitle{思想}
生物免疫系统是一个分布式、自组织和具有动态平衡能力的自适应复杂系统。
它对外界入侵的抗原,可由分布全身的不同种类的淋巴细胞产生相应的抗体,
其目标是尽可能保证整个生物系统的基本生理功能得到正常运转。

~

具有较强模式分类能力,尤其对多模态问题的分析、处理和求解表现出较高的智能性和鲁棒性。
}

\frame{\frametitle{思想}
	\begin{itemize}
		\item<1-> 抗原和抗体
		\item<2-> 免疫疫苗
		\item<3-> 免疫算子
		\item<4-> 免疫调节
		\item<5-> 免疫记忆
		\item<6-> 抗原识别
	\end{itemize}
}

\frame{\frametitle{种类}
	\begin{itemize}
		\item<1-> 一般免疫算法
		\item<2-> 阴性选择和克隆选择算法
		\item<3-> 免疫网络学说与人工免疫网络模型
		\item<4-> 混合免疫算法
	\end{itemize}
}

\frame{\frametitle{发展}
在基于马尔科夫链的收敛性分析和非线性动力学模型等方面,
对免疫优化算法的非线性随机分析可能是未来研究的难点之一。

~

神经网络、内分泌及免疫这三大调节系统相互联系、相互补充和配合、相互制约的机理
为基于人工免疫系统的智能综合集成提供了生物学基础,
网络和智能成为免疫算法发展的不可缺少的特征,也是其重要应用领域。

~

免疫算法能增强系统的鲁棒性,在网络、智能系统和鲁棒系统中的应用。
}

%end
