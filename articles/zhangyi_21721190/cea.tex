% author:         zhangyi zju_cs
% environment:    ubuntu 16.04
%                 textlive-full
%                 tex-xetex
% editor:         vscode
%                 latex-workshop
% compiler:       xelatex
% course:         生物智能与算法
% teacher:        yuanxi

\documentclass[a4paper]{article}

\usepackage{ctex}       % support chinese
\usepackage{geometry}   % setting margin
\usepackage{setspace}   % setting space


\title{多种群协同进化}
\date{2018-04-30}
\author{张毅}

\geometry{left=3.5cm,right=3.5cm,top=4.5cm,bottom=4.5cm}

\begin{document}
    
    \doublespacing
    \pagenumbering{gobble}
    \maketitle
    \newpage
    \pagenumbering{arabic}

    \tableofcontents
    \newpage

    \section{背景}

    协同演化算法(coevolutionary algorithm, CEA)针对演化算法的不足而兴起.它模拟生物协同演化的思想,通过构造两个或者多个种群,建立他们之间的竞争或者合作关系,使多个种群通过相互作用来提高各自的性能,适应复杂系统的动态演化环境,以达到种群优化的目的.

    与协同演化算法相比,演化算法只采用了基于个体自身适应度的演化模式,没有考虑其演化的环境与个体之间的复杂联系,它在应用中表现出了易出现未成熟收敛且收敛速度慢的问题.协同演化算法可以利用少数起演化导向作用的个体,减少不必要的计算量,使收敛速度加快.


\end{document}