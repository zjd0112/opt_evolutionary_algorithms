%
% GNU courseware, XIN YUAN, 2017
%

\section{复杂网络}

\frame{
\centerline{\textbf{\Huge{复杂网络}}}
}

\frame{\frametitle{定义}

具有自组织、自相似、吸引子、小世界、无标度中部分或全部性质的网络称为复杂网络。
}

\frame{\frametitle{思想}

互联网拓扑、社交网络、脑神经网络、基因调控网络、蛋白质调控网络、生态系统、地球演化、N体作用。
}

\frame{\frametitle{思想}
	\begin{itemize}
		\item<1-> 结构复杂:节点数巨大,结构具有多种不同特征。
		\item<2-> 网络进化:表现在节点或连接的产生与消失,网络结构不断变化。
		\item<3-> 连接多样性:节点之间的连接权重存在差异,且有可能存在方向性。
		\item<4-> 动力学复杂性:节点集可能属于非线性动力学系统,节点状态随时间发生复杂变化。
		\item<5-> 节点多样性:节点和连接可代表任何事物。
		\item<6-> 多重复杂性融合:多重复杂性相互影响,结果难以预测。
	\end{itemize}
}

\frame{\frametitle{思想}
网络的几何性质,网络的形成机制,网络演化的统计规律,网络上的模型性质,
网络的结构稳定性,网络的演化动力学机制。
}

\frame{\frametitle{种类}
	\begin{itemize}
		\item<1-> 马尔科夫场
		\item<2-> 贝叶斯网络
		\item<3-> 社交网络
		\item<4-> 视觉听觉网络
		\item<5-> 大脑生理病理网络
		\item<6-> 组分调控网络
		\item<7-> 生态系统网络(含传染病网络)
		\item<8-> 网络安全攻防
		\item<9-> 天文N体系统
	\end{itemize}
}

%end
