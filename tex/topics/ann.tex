%
% GNU courseware, XIN YUAN, 2017
%

\section{人工神经网络}

\frame{
\centerline{\textbf{\Huge{人工神经网络}}}
}

\frame{\frametitle{定义}

人工神经网络是由大量处理单元互联组成的非线性、自适应信息处理系统。
}

\frame{\frametitle{思想}

在现代神经科学研究成果的基础上提出,试图通过模拟大脑神经网络处理、记忆信息的方式进行信息处理。

~

非线性、非局限性、非定常性、非凸性。

~

自适应、自组织、自学习。
}

\frame{\frametitle{思想}

基于逻辑符号和规则推理的专家系统,在处理直觉、非结构化信息方面有缺陷。

~

人工神经网络是连接主义的观点,是神经元相互联接而成的自适应非线性动态系统,
分为前向网络(有向无环图)和反馈网络(无向完备图)两类。

~

理论研究:MP模型、Hebb规则、感知器、适应谐振ART理论、自组织映射、认知机网络、非线性动力学。
}

\frame{\frametitle{种类}
	\begin{itemize}
		\item<1-> Hopfield
		\item<2-> 波尔兹曼机(模拟退火)
		\item<3-> 联想记忆
		\item<4-> BP
		\item<5-> 局部逼近神经网络(CMAC,RBF,B样条)
		\item<6-> 自组织SOM
		\item<7-> ART
		\item<8-> SVM
		\item<9-> 强化学习
	\end{itemize}
}

\frame{\frametitle{种类}
	\begin{itemize}
		\item<1-> 集成式网络(boosting,梯度推进机,决策树,随机森林)
		\item<2-> 量子神经网络
		\item<3-> 脉冲耦合
		\item<4-> 混沌神经网络
		\item<5-> 深度学习(卷积、循环、递归,置信网络,生成式对抗网络,迁移学习,果蝇的大脑)
	\end{itemize}
}

%end
