%
% GNU courseware, Zhang Jiadong, 2018
%

\section{前言}\fontsize{12pt}{12pt}\selectfont

\frame{\frametitle{定义}
又叫演化计算,是模拟自然界中的生物的演化过程产生的一种群体导向的随机搜索技术和方法。

~

是一种通用的问题求解方法,具有自组织、自适应、自学习性和本质并行性等特点,
不受搜索空间限制性条件的约束,也不需要其它辅助信息。
}

\frame{\frametitle{思想}
进化算法是受生物进化过程中“优胜劣汰”的自然选择机制和遗传信息的传递规律的影响,
通过程序迭代模拟这一过程,把要解决的问题看作环境,在一些可能的解组成的种群中,
通过自然演化寻求最优解。
}

\frame{\frametitle{种类}
	\begin{itemize}
		\item<1-> 遗传算法(Genetic Algorithms)
		\item<2-> 演化策略(Evolution Strategy)
		\item<3-> 演化规划(Evolution Programming)
		\item<4-> 遗传程序设计(Genetic Programming)
		\item<5-> 多种群协同进化(Coevolutionary Algorithm)
		\item<6-> 差分进化算法(Differential Evolutionary)
	\end{itemize}
}

%end
